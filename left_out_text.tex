

%\notebyalbert{No comprendo esta frase:} \emph{This classification of types of lines automatically leads to specific regions arranged along all longitudes associated with different thermodynamic, with good symmetry between both hemispheres in both rotations.}

----------------------------------------------------

%\temp{mostrar los gradientes negativos va a ser util en los parrafos posteriores que expliquen porque en las lineas tipo 0 hay valores de $\phi_c$ negativos.} 

%\temp{ Comentar en la discusion el paper de shift y cramer, para decir un poco sobre los down y la incapacidad del modelo en reproducirlos}


----------------------------------------------------

%To establish if the differences in the results between the two rotations are significant, Section \ref{uncertainties} estimates the error bars due the dominating systematic uncertainties.
%\diego{quizas vale la pena introducir alguna comparacion sobre el campo magnetico, veremos como quedan los comentarios de ceci.}
%{Also, within the streamer belt the magnetic field strength is similar in both rotations in the northern hemisphere, while in the southern hemisphere CR-1915 exhibits $10-80\%$ larger values compared to CR-2081, with increasing difference for larger latitudes. In the coronal holes, the magnetic field strength of CR-1915 is much larger than for CR-2081, exhibiting values up to 50\% larger in the northern hemisphere and more than twice larger in the southern hemisphere.}

%Previous works have found magnetic loops which temperature decrease with height, most generally in equatorial latitudes \citep{huang_2012,nuevo_2013}. This type of loops with negative gradients produce a negative conductive flux that contribute to the heating in the corona. In this work, we compute the loop-integrated energy quantities separately in three different populations of loops. 

----------------------------------------------------

%Type 0 loops present slightly larger gradients than type I. Therefore, type 0 present bigger amount, in modulus, of loop-integrated conductive flux. Due conductive flux negative can not be assumed as a loss but it can be contributing to the coronal heating. Thus, to compensate the radiative losses, a lower loop-integrated energy input flux ($\phi_h$) is required.

%We can see, on type 0 loops, a median value of energy input flux for CR2082 twice bigger than CR2208. Also, in both rotations, we observe a negative population of loop integrated energy input flux. This can refer to some loops where loop-integrated radiative flux is too small in comparison with the larger gradients of negative conductive flux, resulting in negative energy input flux. 

%The type I loops show a loop-integrated positive conductive for each loop in both rotations consistent with positive gradients of temperatures. The conductive flux have values around 0.3 $10^{5}[\rm{erg\,sec^{-1}\,cm^{-2}}]$ in both rotations. Similar results have been found for loop-integrated radiative flux and energy input flux for both rotations.

----------------------------------------------------

 with latitudinal gradients of both density and temperature  maximum in the open-close boundary. The open field regions in low latitudes maintain thermodynamic values associated with open regions.} The reconstruction of CR-2082 does not show clear the ARs, while CR-2208 shows a well reconstruction of both AR around latitude-longitud $+5\mdeg$, $140\mdeg$ and $+5\mdeg$, $300\mdeg$. In both rotations the open-closed boundaries derived from the AWSoM model match the shape of the contour levels of both the electron density and the electron temperature.
 
 
 
